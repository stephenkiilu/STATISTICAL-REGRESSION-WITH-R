\documentclass[11pt]{beamer}
\usetheme{CambridgeUS}
\usepackage[utf8]{inputenc}
\usepackage[english]{babel}
\usepackage{amsmath}
\usepackage{amsfonts}
\usepackage{amssymb}
\usepackage{float}
\usepackage{graphicx}
\title{Application of Poisson regression in modeling insurance claims in Singapore}
\author{Stephen Kiilu\\ Supervised by\\ Prof.\,Viani Djeundje Biatat}
%\institute{African Institute for Mathematical Sciences\\ Rwanda}
\titlegraphic{\includegraphics[height=1.5cm]{aims}}
%\setbeamercovered{transparent} 
%\setbeamertemplate{navigation symbols}{} 
%\logo{}  
%\date{} 
%\subject{} 
\begin{document}

\begin{frame}
\titlepage
\end{frame}

%\begin{frame}
%\tableofcontents
%\end{frame}

\begin{frame}{Contents}
\begin{itemize}
\item[•]Introduction
\item[•]Model building
\item[•]Output interpretation
\item[•]Confidence intervals
\item[•]Odd ratios
\item[•]Model fit
\item[•]Improving the model
\end{itemize}
\end{frame}
\begin{frame}{Introduction}
My main task is to work with SingaporeAuto (General Insurance Association of Singapore) data set and use regression model to determine main drivers of insurance claims.
The data set provided
\begin{table}[H]
\centering
\begin{tabular}{rlrlrrrrrr}
  \hline
 & Sex & Gender & VType & PC & Clm & Exp & LNW & NCD & AgeCat \\ 
  \hline
1 & U &   0 & T &   0 &   0 & 0.67 & -0.40 &  30 &   0 \\ 
  2 & U &   0 & T &   0 &   0 & 0.57 & -0.57 &  30 &   0 \\ 
  3 & U &   0 & T &   0 &   0 & 0.50 & -0.69 &  30 &   0 \\ 
  4 & U &   0 & T &   0 &   0 & 0.91 & -0.09 &  20 &   0 \\ 
  5 & U &   0 & T &   0 &   0 & 0.54 & -0.62 &  20 &   0 \\ 
  6 & U &   0 & T &   0 &   0 & 0.75 & -0.28 &  20 &   0 \\ 
   \hline
\end{tabular}
\end{table}
\end{frame}
\begin{frame}{Descriptive analysis}
\begin{figure}[H]
\includegraphics[width=6cm]{R1}
\centering
\caption{Descriptive analysis}
\end{figure}
\end{frame}
\begin{frame}{Descriptive analysis cont.}
\begin{figure}[H]
\includegraphics[width=7cm]{R2}
\centering
\caption{Descriptive analysis}
\end{figure}
\end{frame}
\begin{frame}{Model building}
\begin{table}[H]
\centering
\begin{tabular}{rrrrrr}
  \hline
 & Estimate & Std. Error & z value & Pr($>$$|$z$|$) \\ 
  \hline
(Intercept) & -1.9086 & 0.0913 & -20.91 & 0.0000&$^{***}$ \\ 
  NCD10 & -0.2816 & 0.1247 & -2.26 & 0.0239&$^{*}$ \\ 
  NCD20 & -0.4705 & 0.1274 & -3.69 & 0.0002&$^{***}$ \\ 
  NCD30 & -0.3680 & 0.1947 & -1.89 & 0.0588 &$^{.}$\\ 
  NCD40 & -0.7255 & 0.2436 & -2.98 & 0.0029&$^{**}$ \\ 
  NCD50 & -0.7014 & 0.1428 & -4.91 & 0.0000 &$^{***}$\\ 
  Age2 & 0.3064 & 0.3110 & 0.98 & 0.3247 &$^{}$\\ 
  Age3 & 0.4610 & 0.1169 & 3.94 & 0.0001&$^{***}$ \\ 
  Age4 & 0.4089 & 0.1307 & 3.13 & 0.0018&$^{**}$ \\ 
  Age5 & 0.2389 & 0.1972 & 1.21 & 0.2256 \\ 
  Age6 & 0.7595 & 0.2595 & 2.93 & 0.0034&$^{**}$ \\ 
  Age7 & 0.8584 & 0.7166 & 1.20 & 0.2310 \\ 
   \hline
\end{tabular}
\end{table}
\end{frame}
\begin{frame}
(Dispersion parameter for poisson family taken to be 1)

    Null deviance: 2716.9  on 7482  degrees of freedom
Residual deviance: 2664.6  on 7471  degrees of freedom \\
AIC: 3683.8

Number of Fisher Scoring iterations: 6
\end{frame}
\begin{frame}{Interpretation}
\begin{itemize}
\item[•]Having a NCD10 versus NCD0, the log odds for number of claims changes by -0.2816
\item[•]Having a NCD50 versus NCD0, the log odds for number of claims changes by -0.7014
\item[•]Comparing age category one and age category two, the log odds for the  number of claims increases by 0.3064
\item[•]Positive coefficient means that as the covariate increase, also the number of claims in a year increase
\end{itemize}

\end{frame}
\begin{frame}{Confidence intervals}
In logistic models, the confidence intervals can be obtained using log likelihood functions, but we can also get them based on the standard errors.
\begin{table}[H]
\centering
\begin{tabular}{rrrr}
  \hline
 & Estimate & 2.5 \% & 97.5 \% \\ 
  \hline
(Intercept) & -1.91 & -2.09 & -1.73 \\ 
  NCD10 & -0.28 & -0.53 & -0.04 \\ 
  NCD20 & -0.47 & -0.72 & -0.22 \\ 
  NCD30 & -0.37 & -0.77 & -0.00 \\ 
  NCD40 & -0.73 & -1.24 & -0.28 \\ 
  NCD50 & -0.70 & -0.99 & -0.42 \\ 
  Age2 & 0.31 & -0.36 & 0.87 \\ 
  Age3 & 0.46 & 0.23 & 0.69 \\ 
  Age4 & 0.41 & 0.15 & 0.66 \\ 
  Age5 & 0.24 & -0.16 & 0.61 \\ 
  Age6 & 0.76 & 0.21 & 1.24 \\ 
  Age7 & 0.86 & -0.95 & 2.01 \\ 
   \hline
\end{tabular}
\end{table}
\end{frame}
\begin{frame}{Odds-ratio}
Often we use odds ratios to interpret logistic models, by just using the same logic as with log odds.
\begin{table}[ht]
\centering
\begin{tabular}{rrrr}
  \hline
 & OR & 2.5 \% & 97.5 \% \\ 
  \hline
(Intercept) & 0.15 & 0.12 & 0.18 \\ 
  NCD10 & 0.75 & 0.59 & 0.96 \\ 
  NCD20 & 0.62 & 0.48 & 0.80 \\ 
  NCD30 & 0.69 & 0.46 & 1.00 \\ 
  NCD40 & 0.48 & 0.29 & 0.76 \\ 
  NCD50 & 0.50 & 0.37 & 0.65 \\ 
  Age2 & 1.36 & 0.69 & 2.38 \\ 
  Age3 & 1.59 & 1.26 & 1.99 \\ 
  Age4 & 1.51 & 1.16 & 1.94 \\ 
  Age5 & 1.27 & 0.85 & 1.84 \\ 
  Age6 & 2.14 & 1.24 & 3.45 \\ 
  Age7 & 2.36 & 0.39 & 7.49 \\ 
   \hline
\end{tabular}
\end{table}
\end{frame}
\begin{frame}
\begin{itemize}
\item[•]Holding other predictor variables constant, the odds for the number of claims for NCD10 over NCD0 is 0.75 times i.e it is lower as compared to our reference NCD0
\item[•]The odds for age category seven is 136 \%  higher compared to odd for age category 1.
\item[•]The log odd ranges from 0 to 1, while odd ratios range from 1 to infinity. As log odds increase, the odd ratios increase. We use 1 as bas to interpret odd-ratios.
\end{itemize}
\end{frame}
\begin{frame}{Model fit\,-\,Likelihood ratio test}
Often we may want to measure how our model fits. The test is too see if the predictors fits well compared to a null model. The test is the difference between deviance between two models.
For this model the chi-square of 52.24953, with 11 degrees of freedom and a p-value $<$0.0001. This means our model fits significantly better than a null model.
\end{frame}
\begin{frame}{Improving model}
The goal is to have a good model that signficantly fits our data. Some of ways to improve the model include: adding relevant  covariates, checking outliers, missing data and problem of  overdispersion.
\end{frame}
\begin{frame}
\centering
\huge{END.}\\
THANK YOU!!
\end{frame}
\cite{newtest}
\bibliography{s} 
\bibliographystyle{ieeetr}
\end{document}